\documentclass[12pt]{amsart}

\usepackage{amsfonts,latexsym,amsthm,amssymb,amsmath,amscd,euscript,graphicx,tikz}
\usepackage{framed}
\usepackage{fullpage}
\usepackage{hyperref}
    \hypersetup{colorlinks=true,citecolor=blue,urlcolor =black,linkbordercolor={1 0 0}}

\newenvironment{statement}[1]{\bf\smallskip\noindent\color[rgb]{1.0,0,.0} {\bf #1.}}{}
\allowdisplaybreaks[1]

\newtheorem{theorem}{Theorem}
\newtheorem*{theorem*}{Theorem}
\newtheorem*{proposition}{Proposition}
\newtheorem{lemma}[theorem]{Lemma}
\newtheorem{corollary}[theorem]{Corollary}
\newtheorem{conjecture}[theorem]{Conjecture}
\newtheorem{postulate}[theorem]{Postulate}
\theoremstyle{definition}
\newtheorem{defn}[theorem]{Definition}
\newtheorem{example}[theorem]{Example}

\theoremstyle{remark}
\newtheorem*{remark}{Remark}
\newtheorem*{notation}{Notation}
\newtheorem*{note}{Note}

\newcommand{\BR}{\mathbb R}
\newcommand{\BC}{\mathbb C}
\newcommand{\BF}{\mathbb F}
\newcommand{\BQ}{\mathbb Q}
\newcommand{\BZ}{\mathbb Z}
\newcommand{\BN}{\mathbb N}

\newcommand{\nc}{\newcommand}

\nc{\on}{\operatorname}
\nc{\Spec}{\on{Spec}}

\title{Classical and Quantum Waves Brief} 
\date{\today}

\begin{document}

\maketitle

\vspace*{-0.25in}
\centerline{Vincent Cai}
\centerline{\today}
\vspace*{0.15in}

\section{Loaded String System}

Typically a system of coupled harmonic oscillators can be solved by finding the normal modes
of the system, which are solutions where all components of the system oscillate at the same 
frequency and in the same phase as each other. Then, the complete solution of the system can 
be found by considering the space of all linear combinations of the normal modes. \\ 
\\ 
Mathematically, a normal mode is a set of equations, each describing a classical object of the 
system in queston where each equation is of the form $Ae^{i(\omega t + \delta)}$, with $A, \delta, \omega \in \BR$, for a fixed 
$\omega$ and $\delta$, as these represent the frequency and phase of the normal mode, respectively (Note 
that we also restrict $A$ to be real since a complex amplitude will introduce a phase). Thus, in a 
system of coupled harmonic oscillators, where the equations of motion of a system can typically all 
be written in the form $a\ddot x_i + b \dot x_j = c x_k$, for $a,b,c \in \BC$ (usually real system don't have complex coefficients in the equations of motion, but mathematically it does generalize), where $x_i$ is the position element of the $i$th classical object in the system (this is the defining assumption that 
characterizes a "couple oscillator system" for us here, with a potential for damping expressed by the first 
derivative term of position), we can always reduce it down to a linear system of equations by substituting $x_i = A_ie^{i\omega t + \delta}$, and
obtaining $-\omega^2aA_i + i\omega bA_j = cA_k$. All we don't know now is $\omega$. \\ 
\\ 
Now we argue that a loaded string, under certain approximations, can be mathematically represented by this type of system. Consider a system of a series of $n$ masses of mass $m$, each 
equidistant from each other by distance $l$, and each connected by a string with constant tension $T$, where the two masses at the end are attached to $y=0$. The setup is shown in the figure below:\\
\\
\begin{center}
    \includegraphics[scale=.6]{images/loadedstring.JPG}
\end{center}
\begin{center}
    \bf{Figure 1}
\end{center}
\vspace{.25in}
We exclusively consider the y-coordinate of each mass, using the approximation that the x-position of each mass stays roughly constant, as held together by opposing forces from the 
tension in the string. \\ 
\\ 
Next, we make the assumption that the y-dispaclement is much larger than the separation between any two masses. This allows us to use the small 
angle approximation and say that $tan(\theta) \approxeq sin(\theta)$.\\
\\ 
Thus, for some mass $i$, the equations of motion are: 

\begin{align*} 
    F &= ma \\ 
    T_{left} + T_{right} &= m\ddot y_i \\ 
    -Tsin(\theta_{left}) + Tsin(\theta_{right}) &= m\ddot y_i \\ 
    -Ttan(\theta_{left}) + Ttan(\theta_{right}) &= m\ddot y_i \\ 
    T \frac{y_{i-1}-y_i}{l} + T \frac{y_{i+1}-y_i}{l} &= m\ddot y_i \\
    y_{i-1} - 2y_i +y_{i+1} &= \frac{ml}{T}\ddot y_i
\end{align*}

On the boundary, at $i = 1$ and $i = n$, we take $y_0 = y_{n+1} = 0$, since 
the strings at the end attach to $y = 0$.\\ 
\\ 
Now this is a linear system of 2nd order differential equations precisely in the form 
that we use to solve coupled oscillator systems. Therefore, we can solve for normal modes, 
where $y_i(t) = A_i e^{i\omega t + \delta}$, so our differential equations become algebraic 
equations in the form $y_{i-1} - 2y_i + y_{i+1} = -\omega ^2 \frac{ml}{T}\ddot y_i$.\\ 
\\ 
This system can written as the following matrix equation:

\begin{center}
    \[
        \begin{bmatrix}
            2 & -1 & 0 &  \dots &&0 \\ 
            -1 & 2 & -1 & 0 & \dots \\ 
            0 & -1 & 2 & -1 & 0 &  \\ 
            \vdots & 0 & \ddots & \ddots & \ddots & \vdots      
        \end{bmatrix}
        \begin{bmatrix}
            y_0 \\
            \vdots \\
            \vdots \\
            y_n
        \end{bmatrix}
        = 
        \omega^2 \frac{ml}{T}
        \begin{bmatrix}
            y_0 \\
            \vdots \\
            \vdots \\
            y_n
        \end{bmatrix}
    \]
\end{center}
\vspace{.25in}
This is an eigenvalue problem for $\omega$. We can see from this that since the left tridiagonal matrix 
is it's own conjugate transpose, it is Hermitian (or self-adjoint), which implies there exists an orthonormal eigenbasis of 
this operator, which therefore implies that there are n distinct 1-dimensional eigenspaces of this operator. Therefore, \textbf{a loaded string 
of $n$ masses will have $n$ distinct normal modes}, since these eigenspaces will span an $n$ dimensional vector space. 

\section{Numerical Solutions to Schrodinger Equation}

We consider the one-dimensional time-independent Schrodinger Equation for a potential $V(x)$. Our goal is to 
generate numerical solution to potentials that are difficult to analytically solve for. To do this, we need to 
discretize our Hilbert Space, since we do not have infinite computing power. We approach this by "sampling" values
of our wavefunction in discrete intervals, forming a discrete basis  $\{x_0, \dots, x_n\}$ for a new vector space that we can construct our 
eigenvalue problem for.\\ 
\\ 
We then need to construct our Hamiltonian in this discrete basis. For $\hat H = -\frac{\hbar^2}{2m} \frac{\partial}{\partial^2x} + V(x)$, the potential term 
is simple, for it is simply a diagonal matrix with the $i$th diagonal entry as $V(x_i)$ (represents multiplying each position value by $V(x)$). The second derivative 
at $x_i$ is approximately $\frac{x_{i-1}-2x_{i}+x_{i+1}}{2dx}$, so our Hamiltonian in this new finite-dimensional sampled vector space is:

    \[\bar H = 
    \frac{\hbar^2}{2m \times dx}  
    \begin{bmatrix}
        2 & -1 & 0 &  \dots &&0 \\ 
        -1 & 2 & -1 & 0 & \dots \\ 
        0 & -1 & 2 & -1 & 0 &  \\ 
        \vdots & 0 & \ddots & \ddots & \ddots & \vdots      
    \end{bmatrix} + \begin{bmatrix}
        V(x_0) & \dots & & & 0 \\
        \vdots & V(x_1) & \dots& \\
        &\vdots&\ddots &&\\
        0 &&&0 
    \end{bmatrix} \]
\vspace{.25in}\\
\\
For this Hamiltonian, we are assuming that $x_0 = x_n = 0$, which is necessary to get a square matrix to represent an operator on a vector space, 
but introduces boundary conditions that an infinite potential well at both endpoints would introduce. Thus, all solutions using this numerical method 
model those in an infinite potential well.\\
\\
Now we can solve the Hamiltonian eigenvalue problem in this discrete position-basis vector space for a vector of position values $\{x_0,  \dots , x_n\}$, which 
can give us a sample of our wavefunction solutions. Additionally, when this is implemented in code, $\hbar$ is set to a constant, usually $1$, so all units are in 
terms of $\hbar$. \\ 
\\ 



\end{document}